\documentclass[titlepage]{scrartcl}
\usepackage{enumitem}
\usepackage[british]{babel}
\usepackage[style=apa, backend=biber]{biblatex}
\DeclareLanguageMapping{british}{british-apa}
\usepackage{url}
\usepackage{float}
\usepackage{caption}
\restylefloat{table}
\usepackage{perpage}
\MakePerPage{footnote}
\usepackage{abstract}
\usepackage{graphicx}
% Create hyperlinks in bibliography
\usepackage{hyperref}
\usepackage{amsmath}

\usepackage[T1]{fontenc}
\usepackage[utf8]{inputenc}
\usepackage{blindtext}
\setkomafont{disposition}{\normalfont\bfseries}

\graphicspath{
    {./resources/},
}
\addbibresource{ ~/Documents/library.bib }

\newsavebox{\abstractbox}
\renewenvironment{abstract}
  {\begin{lrbox}{0}\begin{minipage}{\textwidth}
   \begin{center}\normalfont\sectfont\abstractname\end{center}\quotation}
  {\endquotation\end{minipage}\end{lrbox}%
   \global\setbox\abstractbox=\box0 }

\usepackage{etoolbox}
\makeatletter
\expandafter\patchcmd\csname\string\maketitle\endcsname
  {\vskip\z@\@plus3fill}
  {\vskip\z@\@plus2fill\box\abstractbox\vskip\z@\@plus1fill}
  {}{}
\makeatother

\DeclareCiteCommand{\citeyearpar}
    {}
    {\mkbibparens{\bibhyperref{\printdate}}}
    {\multicitedelim}
    {}

% MATLAB Code block stuff...
\usepackage{color}
\usepackage{listings}

\definecolor{dkgreen}{rgb}{0,0.6,0}
\definecolor{gray}{rgb}{0.5,0.5,0.5}

\lstset{language=Matlab,
   keywords={break,case,catch,continue,else,elseif,end,for,function,
      global,if,otherwise,persistent,return,switch,try,while},
   basicstyle=\ttfamily,
   keywordstyle=\color{blue},
   commentstyle=\color{gray},
   stringstyle=\color{dkgreen},
   numbers=left,
   numberstyle=\tiny\color{gray},
   stepnumber=1,
   numbersep=10pt,
   backgroundcolor=\color{white},
   tabsize=4,
   showspaces=false,
   showstringspaces=false}

\begin{document}
\title{ECS732U --- Real-time DSP}
\subtitle{\LARGE{Assignment 2 Report}}
\author{Sam Perry --- EC16039}

\maketitle

\section{Design process}
Due to the detailed instructions of this assignment, the design process for the
majority of the project, aside from the bonus features, was largely a process
of following the instructions provided. Sections would be added and tested
using the \texttt{gdb} debugger to check that they worked as expected. The most
significant components of the drum machine, such as buttons, the LED and the
accelerometer, were encapsulated in classes to allow for more manageable code.
These classes were then expanded to include additions such as ``debouncers''
for the buttons, and calibration for the accelerometer. A significant amount of
code involved testing for conditions and performing the relevant actions
accordingly. This is detailed in section~\ref{Implementation}. However, the
majority of effort in this project was in the implementation of bonus features
and code that was not detailed specifically in the instructions. This is
detailed in section~\ref{Features}.

\section{Implementation}\label{Implementation}
The project focused on the concept of states, where the drum machine's output
would vary based on it's current state, which could be modified by the user.
For this reason, a significant portion of the code focused on testing
conditions and outputting audio accordingly. This has been illustrated using a
state machine diagram in figure~\ref{RTDSP}. This diagram focuses primarily on
the drum sample playback and shows how the state of components, such as the
button and accelerometer, affected the output of the audio processor. For
example, it can be seen that the drum pattern chosen, as well as it's tempo,
is decided by the state of the accelerometer and potentiometer.

\section{Features}\label{Features}
Figure~\ref{RTDSP} gives a general overview of the audio processing structure
in this project. However, it neglects the lower level details such as the
processing of states within the buttons, accelerometer, LEDs etc\ldots A number of
bonus features were also added to the project.

\subsection{Button de-bouncing}
A \texttt{DebouncedButton} class was added to try and negate the double tap
effect that can occur with the buttons used in this project. This was
implemented by allowing a set period of time to pass after the button had been
pressed where it was assumed to be on. A value of 50ms was used which
effectively removed double-taps, whilst maintaining responsiveness.

\subsection{Accelerometer calibration, reading and mapping}
In order to classify the drum machine as being in one of six orientations, data
from the accelerometer needed to be processed and mapped to create meaningful
output. A calibration function was created to allow samples to be read in for
half a second at the start of the program. The mean of these samples would then
be used in the mapping of the data to provide more accurate classification.
Hysteresis were also implemented to create smooth transitions between
classifications of the six orientations. This accounted for the noise in data
produced by the accelerometer. The result was a class method capable of
reliably outputting an integer relating to each of the six possible orientations.

\subsection{Reverse samples}
Reverse playback of samples was implemented by indexing drum buffers from their
last sample back towards index 0 when the drum machine was positioned upside
down. This was implemented alongside the pattern switching code and allowed for
seamless switching between all patterns in both forward and reverse playback
based on the orientation of the drum machine.

\subsection{Tap detection}
Tap detection was implemented using a high pass filter class written originally
for assignment 1. By filtering out low frequency data, a static threshold could
then be applied to the remaining data to detect sharp increases in Z axis data.
This could then trigger the playing of a fill pattern. This worked effectively,
although this feature may benefit from further tuning of the threshold and
filter cut-off frequency.

\subsection{Mapping tempo to the Y axis}
A second button was added to the project that would re-assign tempo data from
the potentiometer to the Y axis of the accelerometer. This was implemented
through a simple remapping of the Y axis data to an appropriate range and the
setting of the interval using this data as opposed to that of the
potentiometer.

\subsection{Drum sample replacement}
The \texttt{-{}-bonus} argument also replaced the default samples with an
alternative set of samples. This was achieved by modifying the filepath read on
initialisation of the program to point to the location of the alternative
sample in the \texttt{main.cpp} function. 

\newpage
\section{Figures}
\begin{figure}[H]
    \caption{State machine diagram of project}
    \makebox[\textwidth]{\includegraphics[width=1.1\textwidth]{RTDSP}}
    \label{RTDSP}
\end{figure}


\end{document}
